\chapter{FoV Vignette}

One of the most common methods to reduce cybersickness risks and symptoms is decreasing the Field of View \cite{Duh2001}
\cite{Lin2002}.
To alleviate cybersickness during movement with high detail, and movement in the peripheral areas of vision, a
vignette is implemented to limit the Field of View, focusing the users attention and preventing the influence of
activity in the peripheral vision from adding to cybersickness symptoms.
The Vignette is mainly planned for movement close to an object's surface, where peripheral detail is significantly
higher compared to movement in interplanetary space.
The vignette is implemented as a post-processing shader, drawing a 2D effect over the rendered scene based on an
inner and outer radius, which are both adjustable in the settings.
The inner radius determines the maximum distance from the center of the viewport, where a clear field of view is
guaranteed.
While the outer radius determines the minimum distance from the center, after which the screen is fully opaque and
set to a custom color.
The area between the inner and outer radius consists of a gradient, blending between fully transparent, showing the
rendered scene, and the custom color the gradient meets at the outer radius.
Since the vignette is supposed to block peripheral details distracting the user during movement, the vignette is only
drawn during movement, and disabled when standing still or during sporadic movement.
This allows reducing the risk of cybersickness symptoms during critical phases, while still maintaining the feeling
of presence as much as possible, as reducing the field of view negatively influences the feeling of presence \cite{Lin2002}.
An adjustable threshold for the velocity is used, since slow movements tend to only produce low risks of cybersickness
symptoms.
Additionally, an adjustable deadzone is implemented, allowing for a grace period where the vignette is not displayed
when passing the threshold to avoid flickering on short, quick movements, or velocities close to the threshold, that
pass the threshold when fluctuating slightly.
After passing the velocity threshold for at least the deadzone time or longer, the vignette is eased in or out by a
fade animation with an adjustable duration, to make the transition to the limited field of view more comfortable and
less noticeable.
