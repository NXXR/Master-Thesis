In this study, cybersickness, and the most common theories about its origin, as well as popular methods to prevent,
and mitigate resulting symptoms were reviewed.
Based on this information, the CosmoScout VR application has been examined to identify and remedy potential
scenarios and aspects of the application that can cause cybersickness symptoms, with the main focus on navigation in
the virtual environment, especially when CosmoScout VR is used with a VR HMD\@.

The six-degrees-of-freedom navigation is separated into two cases: movement in interplanetary space, and movement
close to, or on a body's surface.
Additionally, the automated navigation was examined and overhauled.

For movement in interplanetary space, it is assumed that the cybersickness symptoms stem from a missing reference
frame, and the control scheme not offering a way to control rotation axes, or translation and rotation, independently.
These factors can lead to disorientation and postural instability, evoking cybersickness symptoms in the user.
To mitigate this, a floor grid has been developed, and implemented to provide the user with a stable reference frame
suggesting an
up-direction along the real world gravity gradient, to help the user maintain postural stability.
In future projects we suggest adapting the control scheme to the floor grid, for example by inverting the movement
controls, to change the interaction context of the simulation from the user being an observer moving through the
solar system simulation, to an exocentric, pseudo-AR  simulation, where the user manipulates the surrounding
simulation using gestures while the observer seemingly remains stationary.
This may further help with cybersickness symptoms according to the sensory conflict theory, as the user is seemingly
not moving through the simulation anymore, invoking less, or no vection.

For free movement close to the surface of a body, it is assumed that the increased peripheral flow during fast
movements causes discomfort.
To remedy this, a field-of-view reducing vignette has been developed, and implemented to reduce peripheral visual flow,
and vection induced in the user.
Several versions of the vignette have been implemented, including a dynamic, and a static vignette, to allow for user
customisation as other studies have shown conflicting results with regard to the optimal FoV limitation.
In future projects, the results of the user study and possible further studies should provide information about the
optimal settings of the vignette, or useful settings to cater to user preferences.
Additionally, potential further work regarding the vignette include adding an option for the vignette to blur the
peripheral areas instead of modifying the opacity, as well as an option to render the vignette as a sphere around the
observer facing the direction of movement, instead of rendering it as a post-processing effect.

The automatic navigation has been identified to cause noticeable cybersickness symptoms during movement.
This is mainly due to the fact, that the automatic movement is based on a linear interpolation between the origin and
target location, and orientation at the same time over the duration of the movement, leading to a mix of translation,
and complex rotation around multiple axes.
To remedy this, a new navigation concept has been developed and major functionality has been implemented.
The new navigation separates the rotation from the translation temporally, providing a more readable and predictable
movement to the user.
Additionally, a structured movement is developed, to facilitate automatic navigation between arbitrary locations
using a flexible set of basic movements to enhance predictability.
In further works, the remaining functions of the new automatic navigation (mainly the transitions between surface and
orbit) should be implemented, including collision detection to modify the navigation path around other bodies that may
be in the way of the automatic navigation.
Finally, the system can also be used to realize more complex movements and different navigation tasks, like guided or
pre-programmed tours between multiple points of interest in the simulation.

While this concludes the possible further works and adjustments to the developed features, the immediate next step is
the realisation of the planned user study to produce insights into the effectiveness of the implemented features.
The user study has been planned in this study, however, the current situation around the COVID-19 pandemic, makes the
execution of a user study more difficult, if not unreasonable.
The execution of the user study also includes the development of the test scenarios, the development of an interface
plugin to allow the subject to submit FMS ratings, and a pre-study to fine-tune the questions used to record subject
satisfaction with the developed features.
The results of the user study are also meant to be used to determine the priority of further development regarding
the implemented features, as well as to offer insights into possible further improvements.
