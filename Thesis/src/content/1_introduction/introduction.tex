Recent advances, and public interest in motion tracking technology (Wii\textcopyright, Kinect\textcopyright), as well
as stereoscopic 3D media and technology have renewed the development in virtual and mixed reality technology.
However, while hardware limitations have advanced, cybersickness is still seen as one of the main detractors to the
widespread adoption of virtual reality technology.
Cybersickness is a form of motion sickness, with symptoms similar to simulator or sea sickness, that is encountered
when experiencing a virtual environment.
Similar to other forms of motion sickness, cybersickness is polysymptomatic and polygenic.
Polysymptomatic means cybersickness can manifest in a wide range of different symptoms and severity, including
nauseagenic symptoms like dryness of mouth or nausea, oculomotor symptoms like eye strain or headache, and
disorientation symptoms like vertigo or dizziness.
Polygenic means symptoms of cybersickness manifest differently for each individual, increasing the difficulty of
comparing cybersickness incidence and severity across groups of users.

While the root causes of cybersickness have not been definitively uncovered, several theories have gained traction
attempting to explain cybersickness occurrence in virtual environments.

Sensory conflict theory attempts to explain cybersickness resulting from conflicting information from the visual and
somatosensory system.
The discrepancy between what the user sees, and the sensory information especially from the vestibular system is
believed to cause symptoms similar to other forms of motion sickness.

Postural instability theory is another popular attempt to explain cybersickness symptoms.
With parallels to the sensory conflict theory, postural instability is believed to be a result of prolonged exposure
to virtual environments and conflicting reference frames.
While the theory does not sufficiently explain the causes of cybersickness through postural instability, they found
instability precedes occurrences of cybersickness.

Measuring cybersickness symptoms can be difficult due to the subjectiveness of many symptoms.
Historically, subjective measurements like questionnaires have been the most popular tool to measure cybersickness
symptoms, their incidence, and severity.
Popular questionnaires to assess cybersickness are the Simulator Sickness Questionnaire (SSQ), as well as a few
similar questionnaires that have been derived from the SSQ with the goal to specialise on virtual environments, as
the SSQ is based on the older motion sickness questionnaire (MSQ) and was created to assess military flight
simulators used by the United States Armed Forces.
The SSQ, and its derived questionnaires are a popular measurement of cybersickness symptoms, and offer different
scales to separate symptoms into categories with sub-scores, allowing to discriminate problem simulators.
However, the SSQ scores are only meant as a comparative measure to the original scores gathered from the military
flight simulators, and as a problem locator unsuitable for general assessment of cybersickness.
Another popular method of subjective cybersickness indication are single-item questionnaires like the Fast Motion
Sickness Scale (FMS) which provides means to measure symptom severity and incidence, however, without separation into
symptom clusters.
Due to the single value measurement, the FMS allows acquisition of measurements at discrete intervals over the whole
exposure period instead of single data points before, and after the exposure.
On the other hand, these measurements provide no insights into the makeup of symptoms that result in the
cybersickness score.

Several objective measurements have been proposed and tested to validate subjective measurements by linking a
physiological response to the subjective data gathered from questionnaires.
Studies have found electrogastrogram (EGG), electrocardiogram (ECG), and respiration measurements to be reliable
indicators for cybersickness symptoms.
While these measurements provide a continuous stream of data over the exposure period, they often require specialised
equipment, and are difficult to interpret correctly.
A more accessible objective measurement is changes in the centre of gravity, indicating postural instability.
Devices to measure postural sway are easier to interpret, and more accessible, as even the inertial measuring unit
(IMU) of VR devices can be used to acquire this data.
However, postural sway cannot be measured continuously over the exposure period, since the movements are difficult to
separate from noise and other confounding factors reliably, and therefore require a specific stance, or period of no
movement to acquire reliable data.

To mitigate the incidence and effects of cybersickness symptoms, several solutions have gained popularity.
First and foremost, there are several best practices that help reduce the severity of cybersickness symptoms, like an
adaptation time for new users, and the avoidance of movements that are either unrealistic in the given context, or
lead to disorienting the user due to the unpredictability, or complexity of movements.
Additionally, limiting the field of view is a popular method to reduce cybersickness by decreasing the peripheral
visual flow.
This leads to less vection, the feeling of self-motion, and less presence, resulting in decreased severity of sensory
conflict.
Another solution, for some cases of cybersickness, is to provide the user with a stable reference frame that provides a
strong indication for the direction of gravity, aiding with postural stability where the virtual environment has
either none or conflicting reference frames for the perceived gravity gradient.

The goal of this thesis is the development of features that prevent and reduce cybersickness for users of the
CosmoScout VR application.
CosmoScout VR is a modular, scientific visualisation of the solar system that allows immersive, exploratory research
on large, multiscale datasets.
In the application, the free and automatic navigation have caused users to experience symptoms of cybersickness and
are therefore the targets of the features developed in this thesis.
Due to the different aspects of cybersickness and the different presumed origins of symptoms in the simulation,
several independent features are developed to combat selected aspects of the virtual environment.
The main separation is the free movement and the automatic navigation.

The free navigation has shown less severe symptoms and require more general solutions.
To combat cybersickness onset during the six-degrees-of-freedom navigation in interplanetary space, a grid is
displayed coinciding with the real world floor to provide a frame of reference for the direction of gravity in an
environment that often lacks a strong indicator of any direction.
Close to the surface of a body, a vignette is provided during movements to reduce peripheral visual flow, and to help
the user focus at the center of the VR device's lenses, reducing vection, and preventing potential eye strain.
To reduce cybersickness during the automatic navigation in the virtual environment, the navigation system was
overhauled to reduce complexity and increase predictability of the movements.
The major change is the separation of rotation, and translation into distinct phases of the movement, and to split the
movement into stages depending on the origin and target location of the movement.
This way, movements can be classified into several general, but flexible navigation scenarios that avoid potentially
provocative movements for the user.

Lastly, a user study is designed to evaluate and validate the effectiveness of the developed features.
The goal of this user study is to compare each developed feature independently to the initial simulation.
During the test scenarios involving free and automatic movement, the subject's cybersickness symptoms and satisfaction
are measured with subjective and objective measurements.
To assess cybersickness symptoms, FMS measurements are planned, to gather discreet symptom data over the exposure
period.
Additionally, postural stability measurements are planned to correlate to the subjective data.
User satisfaction is measured with a questionnaire after each scenario.
The estimated result of the user study is, that automatic navigation will show the most improvement, since the
initial case created noticeable discomfort, and the overhaul of the automatic navigation presents the most
straightforward solution to the cybersickness problems.
The grid and the vignette will likely show less distinct results, since they are generalised solutions to a vague
problem.
The grid may show less effectiveness, as movements in interplanetary space generally produce less sever symptoms, and
the effectiveness may benefit from changes to the control scheme that currently limit the feature's potential.
The field-of-view vignette is a highly individual feature, and disallowing individual customisation in the user study
might prevent testing the feature's full potential in order to reduce confounding factors between subjects.
All in all, it is anticipated that all developed features prove effective means of reducing cybersickness in the
CosmoScout VR application.
