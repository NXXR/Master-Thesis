\section{Motivation}\label{sec:motivation}

Public interest in stereoscopic 3D media and VR technology like the Oculus Rift Kickstarter have renewed the
development in virtual and mixed reality technology.
Mixed reality technology allows for natural interaction with, and immersion in a synthetic environment.
Therefore, it has found use in training, medical treatment, and research.

CosmoScout VR is a modular, scientific visualisation of the solar system that allows immersive, exploratory research
on large, multiscale datasets.
CosmoScout uses a VR environment to create an immersive visualisation, and provide intuitive interfaces to navigate
the large scale simulation.

However, while hardware limitations have advanced, cybersickness is seen as one of the main detractors to the
widespread adoption of virtual reality technology~\cite{Rebenitsch2016}.
CosmoScout VR users have experienced such symptoms of cybersickness, especially during the six Degrees-of-Freedom
movement, and the automatic navigation in the virtual environment.

Cybersickness is a form of motion sickness, with symptoms similar to simulator or sea sickness, that is encountered
when experiencing a virtual environment.
Similar to other forms of motion sickness, cybersickness is polysymptomatic and polygenic.
Polysymptomatic means cybersickness can manifest in a wide range of different symptoms and severity, including
nauseagenic symptoms like dryness of mouth or nausea, oculomotor symptoms like eye strain or headache, and
disorientation symptoms like vertigo or dizziness.
Polygenic means symptoms of cybersickness manifest differently for each individual, increasing the difficulty of
comparing cybersickness incidence and severity across groups of users.


\section{Scope and Goal}\label{sec:scope-and-goal}

The goal of this thesis is the development of features that prevent and reduce cybersickness for users of the
CosmoScout VR application, especially during automatic or free movement through the solar system visualisation.
Due to the different aspects of cybersickness and the different presumed origins of the symptoms in the simulation,
the movement is separated in automatic movement, free movement close to a body's surface, and free movement in
interplanetary space, or distant from other bodies.

The scope of this study includes independent solutions for each case of movement, to combat selected
aspects of the virtual environment, presumed to be the main source of cybersickness symptoms in each situation.
The situations involving free movement have shown less severe symptoms and require more general solutions, while the
automatic movement receives more specific changes.

Collision detection and handling during the automatic movement has been excluded from the scope of this thesis, as
collisions with objects may cause discomfort, or a loss of immersion, but are generally not considered a severe source
of cybersickness symptoms.
However, the changes made to the automatic navigation also have the goal to create a robust system for movements that
allows for the easy, later addition of collision detection and handling, as well as additional automatic movement
scenarios.

Finally, a user study is designed to measure the effectiveness and validity of the developed features, and guide the
further development.
Unfortunately, due to the current situation around the COVID-19 pandemic, the execution of the user study has been
removed from the scope of the thesis.


\section{Overview}\label{sec:overview}

The next chapter (chapter~\ref{ch:background-and-related-work}) of this thesis provides a broad overview over,
Virtual Reality in general, and the CosmoScout application, as well as a review on cybersickness, its symptoms and
causes, as well as popular subjective, and objective methods to measure cybersickness severity, and lastly, common
methods to reduce the impact, or prevent cybersickness symptoms.

After establishing the background and reviewing related works,
chapter~\ref{ch:current-state-and-problems-of-cosmoscout} reviews CosmoScout concepts relevant to this thesis, and
examines the problems of the CosmoScout movement that have been identified to likely cause cybersickness symptoms in
users.

Then, chapter~\ref{ch:implemented-solutions} is separated into three parts, showing the features developed to combat
the identified problems.
The developed features are a floor grid (section~\ref{sec:floor-grid}), a field-of-view vignette
(section~\ref{sec:field-of-view-vignette}), and the rework of the automatic navigation system
(section~\ref{sec:automatic-movement-overhaul}).

Finally, in chapter~\ref{ch:user-study}, the user study to evaluate and validate the new features is designed.
First, the studys goals, methods, and the general concept are presented.
Then, an execution plan is presented, detailing the process and tasks of the study.
Lastly, several hypotheses are established to mark a goal of significant improvement, that we expect the features to
achieve.

The thesis ends with a conclusion of the work, and provides an outlook over potential further work regarding the
cybersickness mitigation in CosmoScout VR\@.
