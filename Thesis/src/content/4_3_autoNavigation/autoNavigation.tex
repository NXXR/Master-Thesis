\section{Automatic Movement Overhaul}\label{sec:automatic-movement-overhaul}

As pointed out in section~\ref{subsec:problems-with-automatic-movement}, the current automatic movement method was
built as temporary means to navigate between saved bookmarks.
The method applies a linear interpolation to transition smoothly between the origin and target position leading to
movement in a straight line between the two points.
Additionally, a linear interpolation between the origin and target orientation is used to rotate the observer into the
target orientation.
\\
While this is the easiest, and most straightforward way to smoothly move the observer through the simulation without
breaking continuity by teleporting the user, the simultaneous translation and rotation have been found to induce
disorientation followed up by motion sickness symptoms by the user.
\\
\\
In this section a new automatic navigation for CosmoScout is presented, which aims to make the automatic navigation
more accessible and less sickness inducing.
As parts of the navigation have not been fully implemented, we first present the concepts how the automatic
navigation system should move the observer.
After the concept, key features of the implemented solutions are presented and explained.


\subsection{Automatic Movement Concept}\label{subsec:automatic-movement-concept}




\subsection{Automatic Movement Implementation}\label{subsec:automatic-movement-implementation}
