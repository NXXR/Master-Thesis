\chapter{Background and Related Work}

While Virtual Reality technology has gained more and more traction over the recent years, 30\% to 80\% of users
encounter some form of sickness symptoms during exposure to virtual reality environments~\cite{Rebenitsch2016}.
Additionally, these sickness symptoms not only occur during exposure, but can have lasting effects and affect users
after the exposure as well~\cite{LaViola2000}.
The high number of affected users has led to cybersickness being one of, if not the biggest roadblock to a more
widespread adoption of Virtual Reality Devices.

According to LaViola~\cite{LaViola2000} the symptoms of exposure to Virtual Reality environments include:
\begin{itemize}
    \item Eye strain
    \item Headache
    \item Pallor
    \item Sweating
    \item Dryness of mouth
    \item Fullness of stomach
    \item Disorientation
    \item Vertigo
    \item Nausea
    \item Vomiting.
\end{itemize}
Vertigo, in the case of VR-sickness specially benign paroxysmal positional vertigo (BPPV), is a condition where the
individual experiences a false sense of motion, or spinning and objects or surroundings appear to swirl or move~\cite{Post2010}.

Throughout the study of these symptoms, several terms have been used to compound these sickness symptoms that appear
to be similar to motion sickness symptoms.
Initially, the term Simulator Sickness was used to describe motion sickness encountered during exposure to flight
simulators and originated from the assessment of military flight simulators~\cite{Saredakis2020, Kennedy1993}.
While Simulator Sickness is still used in recent publications, the terms Cybersickness or VR Sickness are generally used
to differentiate from simulator sickness and closer examine the side effects resulting from the use of virtual
environments~\cite{Saredakis2020,McCauley1992}.
The term VR Sickness specifically is used in discussions and studies about sickness symptoms involving head-mounted
displays (HMD)~\cite{Kim2018,Cobb1999}.
Here, the terms Cybersickness and VR Sickness will be used, as Stanney, Kennedy, and Drexler~\cite{Stanney1997} argue
that sickness from virtual environments shares many of the symptoms often also experienced during simulator sickness
or motion sickness, but the sickness profiles being different.
The main arguments for this distinction are that during cybersickness, disorientation symptoms rank highest and
oculomotor symptoms lowest, while simulator sickness and traditional motion sickness usually have the inverted
profile, where disorientation symptoms rank lowest~\cite{Stanney1997}.
Cybersickness can also occur without stimulation to the vestibular system, purely through visual cues, unlike motion
and simulator sickness, where stimulation of the vestibular system is needed, but not visual stimulation~\cite{LaViola2000}.
