\chapter{User Study}\label{ch:user-study}

Determining the effectiveness of the developed features to reduce cybersickness in the CosmoScout VR application is
difficult since cybersickness is polysymptomatic and polygenic, and can therefore manifest in different ways for each
individual.
This reduces the ability to objectively argue about the effectiveness of the developed features.
All developed features are rooted in positive results for cybersickness reduction in other studies.
However, the interaction context and environment differs from the studies and arguments for the mitigation techniques
in other studies may not be as relevant for the CosmoScout VR environment.
To measure the effectiveness of the developed features, a user study is planned to test the mitigation features on a
sample size of potential users of CosmoScout.

The primary goal of this user study is not to study the effects or causes of cybersickness itself, as this work is
only indirectly aimed at finding the root causes of cybersickness in general or in the CosmoScout VR application
specifically.
Instead, the goal of the study is to implement measures to mitigate the occurrence and impact of cybersickness
symptoms on the user, especially for the navigation in the application.
Therefore, the user study is not conducted to find an amount of cybersickness symptoms that can occur for any
individual, or the specific causes of cybersickness, but to compare the user experience and usability before and
after the implementation of the mitigation features, with a focus on cybersickness symptom incidence.
Additionally, all features were developed targeting a specific aspect of the navigation in the virtual environment
and are not in direct competition with each other.
Therefore, the feature's effectiveness is tested in scenarios tailored to each developed feature, in order to
compare their effectiveness to the original environment without the specific mitigation feature.

While objective measures provide a more direct insight into the occurrence and severity of cybersickness symptoms
during the time subjects spend inside the virtual environment, they often  require special equipment and are
significantly harder to evaluate, and draw conclusions from, correctly.
Physiological measurements as suggested by Kim et al.~\cite{Kim2005} cannot be used for this study, since those
require special medical equipment to measure, and trained personnel to correctly read and interpret.
However, an easier accessible objective measurement is recording the subject's centre of gravity (CoG) either using an
external device like a balance board as proposed by chardonnet et al.~\cite{Chardonnet2015}, or using the VR HMD's
IMU (inertial measuring unit) as employed be Lim et al.~\cite{Lim2020}.
This objective method requires less specialized or no additional equipment, and is easier to read and interpret.
However, as mentioned by Rebenitsch et al.~\cite{Rebenitsch2016}, measuring a subject's CoG requires a specific
stance without movement.
Restricting the subject in this way during the exposure to the virtual environment results either in additional
discomfort for the user, or discontinuous data, diminishing the effectiveness and advantages of objective measurements.
Finally, adding objective measurements to the user study is still beneficial to provide a reference or anchor for the
subjective measures of the user study.

Subjective measures are easier to administer and evaluate, and the user study focuses mainly on these measures and
feedback from the subject's to determine the success of the developed features.
The feedback is also gathered, to guide further improvement of current features, and development of additional
features.
While the Simulator Sickness Questionnaire (SSQ) by Kennedy et al.~\cite{Kennedy1993} is arguably the most popular
subjective measurement to record cybersickness, we agree with recent studies criticising the use of the SSQ in
HMD-based cybersickness research like Sevinc et al.~\cite{Sevinc2020} and Rebenitsch et al.~\cite{Rebenitsch2016}.
These studies dissuade from using the SSQ because of its complex structure and development process that is unsuitable
for modern day HMD-based virtual environments, and diverse users and application types.
Kennedy et al.\ themselves note in their original study, that the SSQ should be used to identify and discriminate
problem simulators, and that the SSQ scores should be used in comparison with the provided calibration sample of
military flight simulators, used by trained military personnel.
Because of these arguments, and its length, we decide not to use the SSQ in our user study to determine the severity
of cybersickness.
For similar reasons, we also discard the questionnaires mentioned in section~\ref{subsec:subjective-measurements}
that are derived from, or based on the SSQ\@.
Finally, we chose to use the Fast Motion Sickness Scale (FMS) developed by Keshavarz et al.~\cite{Keshavarz2011} to
efficiently measure incidence and severity of cybersickness symptoms during the exposure to the virtual environment,
paired with an additional questionnaire in between exposure periods to record user feedback to the mitigation features.
Using the FMS also allows us to sample cybersickness discreetly over the duration of the exposure, and identify
scenarios that are more likely to induce cybersickness, and scenarios where the mitigation features are most effective.


\section{User study concept}\label{sec:user-study-concept}

