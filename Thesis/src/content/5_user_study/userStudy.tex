Determining the effectiveness of the features developed to reduce cybersickness in the CosmoScout VR application is
difficult since cybersickness is polysymptomatic and polygenic, and can therefore manifest in different ways for each
individual.
This reduces the ability to discuss the effectiveness of developed features objectively.
All developed features are based on positive results for cybersickness reduction in other studies.
However, the interaction context and environment differs from these studies and arguments for mitigation techniques
in other studies may not be as relevant for the CosmoScout VR environment.
To measure the effectiveness of the developed features, a user study is planned to test the mitigation features on a
sample size of potential users of CosmoScout VR\@.

The primary goal of this user study is not to study the effects or causes of cybersickness itself, as this work is
only indirectly aimed at finding the root causes of cybersickness in general or in the CosmoScout VR application
specifically.
Instead, the goal of the study is to implement measures to mitigate the occurrence and impact of cybersickness
symptoms on the user, especially for the navigation in the application.
Therefore, the user study is not conducted to identify general incidence or severity scores for cybersickness symptoms
that can occur for any individual, or the specific causes of cybersickness, but to compare the user experience and
usability before, and after the implementation of the mitigation features, with a focus on cybersickness symptom
incidence.
Additionally, all features were developed targeting a specific aspect of the navigation in the virtual environment
and are not in direct competition with each other.
Therefore, each feature effectiveness is tested in scenarios tailored to the developed feature, to compare their
effectiveness to the original environment without the specific mitigation feature.

While objective measures provide a more direct insight into the occurrence and severity of cybersickness symptoms
during the time subjects spend inside the virtual environment, they often require special equipment and are
significantly harder to evaluate, and draw conclusions from, correctly.
Physiological measurements as suggested by Kim et al.~\cite{Kim2005} cannot be used for this study, since those
require special medical equipment to measure, and trained personnel to read and interpret the data correctly.
However, an easier accessible objective measurement is recording the subject's centre of gravity (CoG) either using an
external device like a balance board as proposed by Chardonnet et al.~\cite{Chardonnet2015}, or using the VR HMD's
IMU (inertial measuring unit) as employed be Lim et al.~\cite{Lim2020}.
This objective method requires less specialized or no additional equipment, and is easier to read and interpret.
However, as mentioned by Rebenitsch et al.~\cite{Rebenitsch2016}, measuring a subject's CoG requires a specific
stance without movement.
Restricting the subject in this way during the exposure to the virtual environment results either in additional
discomfort for the user, or discontinuous data, diminishing the effectiveness and advantages of objective measurements.
Finally, adding objective measurements to the user study is still beneficial to provide a reference or anchor for the
subjective measures of the user study.

Subjective measures are easier to administer and evaluate, and the user study focuses mainly on these measures and
feedback from the subject's to determine the success of the developed features.
The feedback is also gathered, to guide further improvement of current features, and development of additional
features.
While the Simulator Sickness Questionnaire (SSQ) by Kennedy et al.~\cite{Kennedy1993} is arguably the most popular
subjective measurement to record cybersickness, we agree with recent studies criticising the use of the SSQ in
HMD-based cybersickness research like Sevinc et al.~\cite{Sevinc2020} and Rebenitsch et al.~\cite{Rebenitsch2016}.
These studies dissuade from using the SSQ because of its complex structure and development process that is unsuitable
for modern day HMD-based virtual environments, and diverse users and application types.
Kennedy et al.\ themselves note in their original study, that the SSQ should be used to identify and discriminate
problem simulators, and that SSQ scores should be used in comparison with the provided calibration sample of
military flight simulators, used by trained military personnel.
Because of these arguments, and its length, we decide not to use the SSQ in our user study to determine the severity
of cybersickness.
For similar reasons, we also discard the questionnaires mentioned in section~\ref{subsec:subjective-measurements}
that are derived from, or based on the SSQ\@.
Finally, we chose to use the Fast Motion Sickness Scale (FMS) developed by Keshavarz et al.~\cite{Keshavarz2011} to
efficiently measure incidence and severity of cybersickness symptoms during the exposure to the virtual environment,
paired with an additional questionnaire in between exposure periods to record user feedback to the mitigation features.
Using the FMS also allows us to sample cybersickness discreetly over the duration of the exposure, and identify
scenarios that are more likely to induce cybersickness, and scenarios where the mitigation features are most effective.

To create a study that, confidently and reliably produces data to determine the effectiveness of the developed
features, a pre-study with a small sample size is recommended in order to fine-tune the process and the
questionnaire items.
Additionally, a pre-study could determine the effectiveness and necessity of CoG or head dispersion measurements, and
whether the correlation between the objective measurements and the FMS scores outweigh the additional time and
inconvenience to collect these measurements.


\section{User study concept}\label{sec:user-study-concept}

The study is designed as a summative, within subjects study, which means every subject is tested on every scenario to
test all developed features with every user, reducing the necessary sample size and the confounding variable of
individual subject preferences.
For within subjects studies, the order of tests can introduce a confounding factor.
However, by counter balancing the test scenarios these effects can be lessened.
The result is a compound study testing each developed feature on its own against the initial scenario without the
feature.
Therefore, the independent variables in each study are the design decisions for the developed features, mainly the
features themselves being turned on or off.
The resulting dependent variables that are examined in this study are slightly different for each feature:
\begin{itemize}
    \item User satisfaction, or user preference for or against each feature
    \item cybersickness incidence and severity for each feature
    \item intuitiveness of controls with the floor grid
    \item obstruction or limitation of usability with the vignette, or the floor grid
    \item predictability of movements for the automatic navigation
    \item task completion time
\end{itemize}
While the task completion time itself is not as important, a significant difference between the scenarios with a
feature turned on or off may indicate problems relating to the other dependent variables.

In order to rely on the results of the study, it is examined according to four quality criteria as suggested
by Preim, and Dachselt~\cite{Preim2015}:

The study's credibility or authenticity is dependent on a clearly defined target group, and the main focus of the study.
The main focus is stated above, to determine the effectiveness of the developed features in reducing cybersickness
symptoms or their severity during a user's exposure to the virtual environment of the CosmoScout VR
simulation.
The target group should be selected to be close to the target user base, while still selecting subjects that are not
directly affiliated with the CosmoScout project, in order to remain unbiased during the study.
Ideally, the selected target group should be as unfamiliar with the project as possible, to reduce possible bias.

The internal validity of the study is given by minimising confounding factors, and a standardised study process.
The process is standardised by providing an execution plan in the next section, including timings for each scenario
and the layout for the questionnaires.
A pre-study should be done with subjects that are not part of the final sample group to fine-tune the time schedule and
questions.
To minimise individual experience with virtual environments or a familiarity with the CosmoScout VR simulation
affecting the results, the execution plan also affords each subject an initial introduction and training time to
become familiar with the simulation's controls and general feel, as the handling of the controls is not of immediate
interest in this study.
Additionally, breaks in between the scenarios are planned to allow the subjects to return towards their baseline to
reduce the confounding factor of time spent inside the simulation, since cybersickness is strongly linked to time
spend inside a virtual environment according to other studies on cybersickenss.

External validity is important to guarantee or indicate the transferability of the results.
Since the developed features are specially targeted at selected problems of the CosmoScout VR application, the
transferability of this study's results is limited.
Some aspects and results of this study may be transferable to similar applications that involve 6-DoF-Navigation and
display similar problems with cybersickness.
To retain external validity, the testing environment, and scenarios are created to resemble the real field of
application.

The reliability of the study is determined by the selection and size of the sample group and its representation of
the real users.
The selection criteria for the subject in the sample group are mentioned above, with the goal to find subjects that
closely resemble the general user base in age and experience with virtual environments.
Since the study is not planned to use concurrent, or group testing of subjects, the study's sample size is based on
subject response, and time spent on data acquisition part of the study.
A longer data acquisition phase allows for a larger sample size.

The primary quality criteria for this study will be the credibility and validity stemming from the study itself, as
the reliability through the sample size has less priority due to the polysymptomatic and polygenic nature of
cybersickness.
While a larger sample size increases the reliability of the results, the individual response of subjects to, and
nature of cybersickness implies that developed solutions have to be individually customisable in order to be efficient.
However, while the developed features are highly configurable to individual needs, allowing the subjects to customise
the features to better fit their needs would introduce a major confounding factor to the study.


\subsection{Execution Plan}\label{subsec:execution-plan}

Initially, basic information about each subject is recorded:
\begin{itemize}
    \item Age
    \item Sex
    \item Experience with VR environments and devices
\end{itemize}
These information are not explicitly used to draw conclusions about the feature effectiveness or cybersickness, but
the study's validity and reliability, indicating the relation of the sample group to the target user group.

After this information is provided, the subject should have around two to five minutes to set up and get used to the
VR HMD and the CosmoScout Virtual Environment, and especially its control scheme.
This training phase should test the subject's familiarity with the control scheme or respectively its intuitiveness.
The training scenario also provides a couple of checkpoints to familiarize the subject with them, so the subject
can recognize them during the later scenarios and identify the task or movement required from them.

Next, the subject is presented with two pairs of two counterbalanced scenarios, two for the floor grid, and two for the
vignette.
Each pair of scenarios consist of one scenario with, and one without the tested mitigation feature.
In both scenarios, the subject has to use the free navigation controls to move through the simulation, and adjust
the observers position and orientation to match each checkpoint.
Once a checkpoint is completed, the next is displayed, and only one checkpoint is displayed in the simulation at any
time during the scenario.
If the checkpoint is outside the observer's view, an indicator is displayed on the side of the viewport, showing the
direction of the next checkpoint on the screen.
The checkpoints are placed in a pattern requiring the subject to perform a mix of easy and complex movements in the
simulation similar to real, and plausible situations.
The checkpoints for the floor grid are placed mainly in interplanetary space, with some closer to a body's surface,
while the vignette checkpoints are placed on or close to the surface of a body with sufficiently detailed surface
textures and elevation data.
The floor grid checkpoints should are placed to show whether the added fixed reference frame of the floor grid helps the
subject to maintain a stable frame of reference and posture.
The vignette checkpoints are placed close to a body's surface to generate sufficient peripheral visual flow.
In this way the checkpoints are placed to test the effectiveness of the developed features based on the reasons and
arguments laid out in the previous chapters of this study (chapters~\ref{subsec:problems-with-free-movement},
~\ref{sec:floor-grid}, and~\ref{sec:field-of-view-vignette}).
Each scenario should contain ten checkpoints, and each scenario pair, should contain the same order of checkpoints to
reliably correlate the FMS scores of both scenarios.
The checkpoints are placed, so that each checkpoint should be accessible within one minute, resulting in the scenario
taking about ten minutes to complete.

At the beginning of each scenario, and after each checkpoint the subject is asked to rate their cybersickness symptoms
on a 20-point scale according to the FMS via a built-in interface plugin with a slider from no symptoms (0) to severe
symptoms (20).
After the initial value, the interface retains the last value to allow the subject to choose their score relative to
the score of the last checkpoint.
In between each scenario, the subject is given a break of four to five minutes outside the virtual environment to
return to their baseline.
Additionally, during the break after each scenario pair, the subject is asked to fill out a questionnaire to rate
their experience.
The Questions are structured as statements that are rated on the Likert scale, and should be fine-tuned in a pre-study.
The initial example questionnaire for the checkpoint scenarios is in the appendix
(figure~\ref{fig:study-checkpoint-questionnaire-1}).
In addition to the questions, a section is provided to allow the subject's to add feedback or other comments
about the test, the feature, or suggestions (figure~\ref{fig:study-checkpoint-questionnaire-2}).
Overall, the four scenarios with ten minutes each, the breaks in between the scenarios, and the initial five minutes for
training should take up roughly one hour of the study.

In the last section of the user study, the automatic navigation is tested in a pair of scenarios.
Each consists of automatic movement between select points of interest covering a variety of navigation scenarios for
the overhauled movement, and some points should be selected to result in potentially provocative movements in the old
automatic navigation to highlight the differences between the old and new navigation.

Both scenario should consist of the same three to five movements between points of interest, resulting in two to five
minutes for each scenario.
After each destination, the user is asked to rate their cybersickness on the FMS, and after each scenario the subject is
given a short break of four to five minutes similar to the other scenarios.
During both breaks the subject is asked to fill out another questionnaire to assess the subject's satisfaction with
the automatic movement, as well as the movement's predictability, with answers on the Likert scale.
An additional question is aimed at finding, whether users prefer short, potentially provocative movements over longer,
less provocative movements.
As with the other questionnaire, a section for feedback, comments, and suggestions is added, and the questions
should be fine-tuned during a pre-study.
The initial example questionnaire is in the appendix (figure~\ref{fig:study-autonav-questionnaire-1}
and~\ref{fig:study-autonav-questionnaire-2}).


\section{Hypothesis}\label{sec:hypothesis}

The study is designed to evaluate the effectiveness of the developed features based on cybersickness incidence and
severity through FMS measurements, and subject acceptance, or preference through questionnaires.
Therefore, we formulate the following hypotheses, where the first two involve the statistical difference between
the cybersickness measurements, as well as the subject's acceptance or satisfaction.
\begin{hypothesis}
    \label{hyp:cybersickness}
    The study shows significant ($p = .05$) results, that each new feature produces less cybersickness incidence and
    severity, with an at least small effect size ($d > 0.2$) between the feature-on and feature-off results.
\end{hypothesis}
\begin{hypothesis}
    \label{hyp:satisfaction}
    The study shows significant ($p = .05$) results, that subject's acceptance or satisfaction is higher for each new
    feature than the featureless version, with an at least small effect size ($d > 0.2$) between the feature-on and
    feature-off results.
\end{hypothesis}

Additionally, we formulate hypotheses about the effectiveness of each feature compared to each other:
\begin{hypothesis}
    \label{hyp:navigation}
    The overhauled automatic navigation will show the most improvement compared to the previous navigation.
\end{hypothesis}
\begin{hypothesis}
    \label{hyp:floor-grid}
    The floor grid will show the least improvement.
\end{hypothesis}
\begin{hypothesis}
    \label{hyp:vignette}
    The vignette will show less improvement than the navigation.
\end{hypothesis}
The reason for Hypothesis~\ref{hyp:floor-grid} is that we estimate general lower cybersickness in interplanetary
space, and the unchanged control scheme might reduce the effectiveness of the floor grid.
We assume the FoV-Vignette (Hypothesis~\ref{hyp:vignette}) will show less effectiveness, because the vignette is not
individually customised to fit each subject, either reducing the effectiveness by being less noticeable, or causing
discomfort or dissatisfaction by being too prominent.

Additionally, we expect the absolute values of the FMS data to be less indicating for the effectiveness of the
features.
However, the rate of change between each checkpoint may indicate movements during the scenario that significantly
improved through the introduction of a feature, or usually generate more cybersickness.
The overall increase of the FMS score over the whole scenario will also provide an indicator for the provocativeness
of the scenario, and the effectiveness of each feature to mitigate the effects of cybersickness over prolonged exposure
to virtual environments.
